%%%%%%weekly meeting template, prepared by Michael Sheng.  09/03/2007
\documentclass[11pt, a4paper]{article}
\usepackage{times}
\usepackage{ifthen}
\usepackage{amsmath}
\usepackage{amssymb}
\usepackage{graphicx}
\usepackage{setspace}

%%% page parameters
\oddsidemargin -0.5 cm
\evensidemargin -0.5 cm
\textwidth 15 cm
\topmargin -1.2 cm
\textheight 22 cm

\renewcommand{\baselinestretch}{1.4}\normalsize
\setlength{\parskip}{0pt}


\begin{document}

%%%mention the no, time, and venue of the meeting
\noindent The {\em first} Software Engineering Group (PG4) Project weekly meeting will be held in {\bf Room 462 (Ingkarni Wardli)} at {\bf 10am on Tuesday 9 August 2016}.


\vspace*{15pt}

\begin{center}
\huge \bf Agenda
\end{center}



%%%first, nominate a chair for the meeting. We suggest that each member at least has one chance as the chair.
\section*{Chair: Tao Zhang}

\vspace*{10pt}

%%%if some students cannot make the meeting due to some reasons, their names should appear here.
\section{Apologies}
\begin{itemize}
\item Yann Frizenschaf
\end{itemize}


%%%short presentation about the work of previous week or any milestone specified in the course.
\section{Presentation}
\begin{itemize}
\item Tao Zhang
\item Yi Lin
\item Yuan Liang
\item Lili Wu
\item Zeqi Fu
\item Navdeep Singh
\end{itemize}

%%%any schedules for this meeting should go next, each with a separate section.
%%%for example, the first meeting is about requirement elicitation, like the following.
\section{Client}
The group shall present the items below with the client:

%%%if there are more subissues, make them as subsections.
\subsection{Requirements Summarise }
Yi Lin shall give a shortly summarise of the client's basic requirements of the SeaFaults Mapping Robot. 

\begin{enumerate}
\item Feature: durable
\item Working Environment: seafloor
\item Function: generated data
\end{enumerate}

\subsection{Requirements collection}
According to the basic requirements team members ask some detailed questions to customers.
\begin{enumerate}
\item Working environment
\begin{itemize}
\item How hazards or debris look like? In which situation the robot should know it is a dangerous site so that it should stop surveying?

\end{itemize}
\item Mapping
\begin{itemize}
\item Do colours map directly to depth?
\item Does the map need to be saved in an image format as well? What format? XML? PDF?
\item How do we deliver the map? As a file?
\item Zoom functionality -- what format? XML? PDF? What is the required resolution/granularity?
\item Is there a maximum size for boundary discontinuities?
\item What is the metric for accuracy of the map?
\item Can the survey function can be turned on and off?
\end{itemize}
\item Movement
\begin{itemize}
\item Is the "robot" autonomous or tethered/controlled?
\item Do we provide the "container"? What is an "appropriate size"?
\item What constitutes "significant force"?
\item If the robot meet any hazard while surveying, what should it do? Stay at that point or moving to the nearest safe place?
\item how long is the robot required to move without pause?
\item how fast should the robot response of our control?
\end{itemize}
\item Operations
\begin{itemize}
\item Are we required to design a UI for the operator? What are the required controls?
\end{itemize}
\end{enumerate}

%%%more issues should make it like the above one.
\section{Other Issues}
Tao Zhang will confirm the next meeting

%%%finally, specifies time of next meeting
\vspace*{10pt}
\noindent Note: Next meeting to be held on 11 August 2016.


\end{document}
