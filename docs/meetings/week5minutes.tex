\documentclass[11pt, a4paper]{article}
\usepackage{times}
\usepackage{ifthen}
\usepackage{amsmath}
\usepackage{amssymb}
\usepackage{graphicx}
\usepackage{setspace}

%%% page parameters
\oddsidemargin -0.5 cm
\evensidemargin -0.5 cm
\textwidth 15 cm
\topmargin -1.2 cm
\textheight 22 cm
\renewcommand{\baselinestretch}{1.4}\normalsize
\setlength{\parskip}{0pt}

\begin{document}


\vspace*{15pt}
\begin{center}
\huge \bf Meeting Minutes
\end{center}

\section*{Chair: Yann Fri}
\vspace*{10pt}

\section{Date and Time}
\begin{itemize}
\item Date: 23/08/2016
\item Time: 10:40am - 11:00am
\end{itemize}

\section{Apologies}
\begin{itemize}
\item none
\end{itemize}

\section{Present}
\begin{itemize}
\item Tao Zhang
\item Yi Lin
\item Yuan Liang
\item Lili Wu
\item Zeqi Fu
\item Navdeep Singh
\item Yann Frizenschaf
\end{itemize}

\section{Secretary}
\begin{itemize}
\item Liang Yuan
\end{itemize}

\section{Summary }
Yann Frizenschaf summarized the progress of the project:

\begin{enumerate}
\item SRS draft is completed, slight changes needed.
\item The overall structure of the project has been defined.
\item Defined basic map data.
\item Sensor research
\end{enumerate}

\section{Basic idea of Map}

\begin{itemize}
\item The map based on grid.
\item The size of each grid is 5cm * 5cm
\end{itemize}

\section{Basic GUI and movement}
\begin{enumerate}
\item GUI
\begin{itemize}
\item Basic GUI is completed.
\item The stop button should be red, it is more clear.
\item All button need to be designed as big as possible.
\item The style of GUI depends on group
\end{itemize}

\item Movement
\begin{itemize}
\item Basic movement, forward. backward, left, right.
\item Emergency stop.
\item Manual operation.
\item Autonomous move to given point.
\end{itemize}
\end{enumerate}

\section{Deliverable}
\begin{itemize}
\item Estimating ability to do the job.
\item Negotiating with client for what would be delivered.
\item According to the feedback of client, provide another deliverable.
\item Ignoring late request.
\end{itemize}



\section{Other Issues}
Do we need to provide exact size of obstacle and fault line?
as exact as possible, the rough size is OK.

\vspace*{10pt}
\noindent Note: Next client meeting to be held on 30 August 2016.


\end{document}