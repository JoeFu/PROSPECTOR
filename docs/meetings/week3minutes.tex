\documentclass[11pt]{article} 



\begin{document}

\title {Meeting Minutes}

\author{Navdeep Singh}

\maketitle



\begin{flushleft}

Date: - 9/08/2016  

\end{flushleft}

\begin{flushleft}

  Time = 10.30 am to 11.00 am 

\end{flushleft}





\begin{flushleft}

\textbf{Client: -} Michael Sheng 

 \end{flushleft}

\begin{flushleft}

\textbf{Attendance: -}

 Navdeep Singh, Tao Zhang, Yi Lin, Yuan Liang, Lili Wu,Zeqi Fu Present 

 \end{flushleft}

 

 \begin{flushleft}

Yann Frizenschaf absent with Apologies

\end{flushleft}

                          



                          \begin{enumerate}

\item What to do when hazard happen?



If there is any obstacle go around it, if there is any no go zone robot cannot go to that zone and team has to find out which one is the no go zone. Also if there is a Fault then don’t step on it.



  \begin{flushleft}

 Suggestion: - Put less hazards                      

\end{flushleft}     











\item Do we have to generate a map or it is going to be provided to us?



Map is contracted out and it’s going to be available in week 6. Before week 6 team has to develop their own map and use it.



\item Obstacle will be some different colour or a 3d thing?



Play with the sensors and find out.



\item No go zone added in beginning or in between the survey.



No go zone can be added in the beginning and also in between.



\item How to detect the fault on a map?



Use colour sensor. But the colour sensor is not accurate. It is hard for sensor to differentiate a dark or light say light grey and dark grey will be same for the sensor. The colour sensor going to scan the colour and will provide a number.



\item Do we need to have a GUI to show the real time survey?



Need to develop a software for the operator to use.Do not use existing GUI of eclipse.

\item Accuracy of the map.



As accurate as possible 



\item Can we control manually or automatically?



Different situation different modes, but implement the manual mode first if it is working correctly then advance feature can be the automatic mode.

\item What will be the Container size? 



Same as the robot size

\item What is significant Force?



Robot should not hit anything. it should not hit the boundary also.

\item How the boundary line will be?

  

  Boundary will be a solid line.

\begin{flushleft}

 \textbf{Note: -} At the end of the course there will be a demonstration for 20 minutes so team should manage their time. Implement the manual working first and then go to automatic if     everything is working correctly but it is not compulsory.

\end{flushleft}

 

\item	The robot will be outside the boundary or inside the boundary.



Always inside the boundary.



\end{enumerate}

 \begin{flushleft}   

Suggestion: -Try all the sensor and try to program and will have more accurate idea about the robot.

\end{flushleft}    



 \begin{flushleft}

\textbf{Note: -} Map Size A1.

\end{flushleft}

 \begin{flushleft}

\textbf{Overall Action: -} Go and make robot working and test all sensors.

\end{flushleft}

 \begin{flushleft}

\textbf{Responsible: -}   Navdeep Singh, Tao Zhang, Yi Lin, Yuan Liang, Lili Wu, Zeqi Fu,Yann Frizenschaf.\end{flushleft}

 \begin{flushleft}

\textbf{Next Meeting: -} 16/8/2016(Tuesday) 10 am

\end{flushleft}







\end{document}
